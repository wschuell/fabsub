\documentclass{beamer}
\usepackage[utf8]{inputenc}
\beamertemplatenavigationsymbolsempty
\begin{document}

\begin{frame}
\begin{center}
\textbf{Tout d'abord, j'aimerais vous parler de mon entourage scientifique,
}
\let\thefootnote\relax\footnotetext{1}
\end{center}
\end{frame}

\begin{frame}
\begin{center}
\textbf{les concepts et idées qui font partie intégrante de mon travail,
}
\let\thefootnote\relax\footnotetext{2}
\end{center}
\end{frame}

\begin{frame}
\begin{center}
\textbf{et qui influencent ma réflexion et mes points de vue sur la robotique.
}
\let\thefootnote\relax\footnotetext{3}
\end{center}
\end{frame}

\begin{frame}
\begin{center}
\textbf{Premièrement, commençons par donner une définition de 'robot'.
}
\let\thefootnote\relax\footnotetext{4}
\end{center}
\end{frame}

\begin{frame}
\begin{center}
\textbf{Les robots sont des systèmes artificiels qui traite les informations provenant de leur capteurs en vue d'auto-produire du mouvement.
}
\let\thefootnote\relax\footnotetext{5}
\end{center}
\end{frame}

\begin{frame}
\begin{center}
\textbf{L'une des chose les plus fondamentale à propos de robots est qu'ils produisent du mouvement, et qu'ils produisent ce mouvement eux-même.
}
\let\thefootnote\relax\footnotetext{6}
\end{center}
\end{frame}

\begin{frame}
\begin{center}
\textbf{Mais cela ne suffit pas.
}
\let\thefootnote\relax\footnotetext{7}
\end{center}
\end{frame}

\begin{frame}
\begin{center}
\textbf{Ce mouvement doit être influencé par les informations venant des capteurs. Les robots ne sont pas de simples executant, ils prennent des décisions.
}
\let\thefootnote\relax\footnotetext{8}
\end{center}
\end{frame}

\begin{frame}
\begin{center}
\textbf{Votre ventilateur n'est pas un robots. Votre ordinateur n'est pas un robot.
}
\let\thefootnote\relax\footnotetext{9}
\end{center}
\end{frame}

\begin{frame}
\begin{center}
\textbf{Mais votre machine à laver l'est peut-être, si elle regule sa vitesse de rotation en la surveillant grâce à un capteur.
}
\let\thefootnote\relax\footnotetext{10}
\end{center}
\end{frame}

\begin{frame}
\begin{center}
\textbf{Ok. Donc c'est la définition d'un robot.
}
\let\thefootnote\relax\footnotetext{11}
\end{center}
\end{frame}

\begin{frame}
\begin{center}
\textbf{Maintenant, je souhaite insister sur le fait qu'il existe des différences fondamentale entre l'apprentissage machine et l'apprentissage chez les robots.
}
\let\thefootnote\relax\footnotetext{12}
\end{center}
\end{frame}

\begin{frame}
\begin{center}
\textbf{En bref, les robots ne peuvent pas apprendre comme les ordinateurs.
}
\let\thefootnote\relax\footnotetext{13}
\end{center}
\end{frame}

\begin{frame}
\begin{center}
\textbf{La robotique developpementale est un champs de recherche extraordinaire.
}
\let\thefootnote\relax\footnotetext{14}
\end{center}
\end{frame}

\begin{frame}
\begin{center}
\textbf{Comme le nom l'indique, elle s'intéresse à l'idée du development chez les robots.
}
\let\thefootnote\relax\footnotetext{15}
\end{center}
\end{frame}

\begin{frame}
\begin{center}
\textbf{L'idée, c'est de donner une enfance aux robots,
}
\let\thefootnote\relax\footnotetext{16}
\end{center}
\end{frame}

\begin{frame}
\begin{center}
\textbf{pendant laquelle ils n'ont pas à être utiles,
}
\let\thefootnote\relax\footnotetext{17}
\end{center}
\end{frame}

\begin{frame}
\begin{center}
\textbf{et au contraire, ils peuvent apprendre et découvrir le monde en s'amusant, comme font les enfants.
}
\let\thefootnote\relax\footnotetext{18}
\end{center}
\end{frame}
\end{document}